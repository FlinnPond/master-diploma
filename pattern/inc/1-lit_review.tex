%%% Литературный обзор %%%
\section{Литературный обзор}

\subsection{Расчёт испарения}

\subsubsection{Поток испарения}

Для того, чтобы измерить потери компонентов при испарении с поверхности сильно разогретой поверхности металла необходимо сформулировать модель испарения и связать параметры на поверности и в атмосфере. 
В данной работе используется модель из статьи \textit{Knight C. J.} \cite{knight1979theoretical}.
В ней поток пара представлется в виде газодинамической задачи Римана о распаде разрыва.  

Картина структуры столба пара представлена на рис. \ref{fig:flowstruct}. Здесь присутствуют обычные компоненты из задачи, такие как ударная волна, контактный разрыв и волна разрежения. Также у основания учитывается наличие слоя Кнудсена, в нём происодит максвеллизация распределения скоростей молекул, вылетевших с поверхности металла. Соотношения для этих переходов выведены Найтом в его статье \cite{knight1979theoretical}.

\addimghere{flowstruct.png}{0.8}{Схематическое представление распределения структуры потока пара над поверностью испарения. В режимах: дозвуковом (слева) и сверхзвуком (справа) 
\cite{klassen2018simulation}.}{fig:flowstruct}

Слой Кнудсена моделируется исходя из предположения, что распеделение скоростей частиц, вытелтающих с поверхности имеет форму половины от распределения максвелла со средней скоростью равной нулю (рис. \ref{fig:pic1-maxwell-freeflow}).

\addimg{pic1-maxwell-contflow.png}{0.7}{Распределение вертикальной компоненты скорости частиц на поверхности и на верхней границе слоя Кнудсена(в паре)}{fig:pic1-maxwell-freeflow}

\clearpage