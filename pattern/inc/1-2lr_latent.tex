%%% Литературный обзор - скрытая теплота перехода %%%

\subsection{Форма бассейна расплава}

Для того, чтобы предсказать состав сплава детали, изготовленной методом селективного спекания металлического порошка, поимимо описания модели испарения необходимо описать поведение вещества в самом бассейне расплава под источником энергии. 

\subsubsection{Скрытая теплота плавления}

Для того, чтобы описать форму бассейна расплава и считать распространение компонентов в нём, нужно описать теплопередачу и фазовый переход между твёрдой и жидкой фазами.
Для этого вводится следующая связь между энтальией и температурой:

\begin{equation}
h=
\begin{cases}
    C_{p,s}T, & T<T_s \\
    C_{p,s}T + h_f\frac{T-T_l}{T_s-T_l}, & T_s <T<T_l \\
    C_{p,l}T + h_f + C_{p,s}(T_s - T_l), & T_l < T
\end{cases}
\end{equation}

Здесь $C_{p,s}$, $C_{p,l}$ -- теплоёмкости в твёрдой и жидкой фазе соответственно, $T_s$, $T_l$ -- крайние температуры полностью твёрдой и жидкой фазы, при которых фазового перехода нет, $h_f$ -- энтальпия фазового перехода.

Такой подход позвоялет моделировать теплоту плавления и более медленный рост температуры во время фазового перехода.

\subsubsection{Конвекция Марангони}

Помимо скрытой теплоты плавления на форму бассейна расплава будет влиять эффекты, связанные с конвекцией. Однако в этой работе не проводится прямого численного моделирования гидродинамики. Поэтому применяется подход, описанный в работе 
\textit{Balbaa, M.; Elbestawi, M.} 
\cite{balbaa2022multi}.
Для учёта конвекции, связанной с эффектом Марангони используется эффективный коэффициент теплопроводности:

\begin{equation}
    k_{eff} = k_l + hL
\end{equation}
где $k_{eff}$ -- эффективный коэффицент теплопроводности, $k_l$ -- коэффициент теплопроводности в неподвижной жидкости, $h$ -- коэффициент конвективной теплопроводности и $L$ -- характерный размер мелтпула, обычно берётся равным половине его ширины.  

Значение коэффициента конвективной теплопроводности внутри жидкого металла может быть рассчитано из числа Нуссельта, используя уравнение \ref{eq:nusselt}, которое определяет соотношение между конвективной и кондуктивной теплопередачей. Поскольку конвекция в жидком металле вызывается эффектом Марангони, число Нуссельта может быть рассчитано как функция числа Марангони, как показано в уравнениях \ref{eq:nusselt_mar} и \ref{eq:marangoni}.

\begin{equation}
    \label{eq:nusselt}
    Nu = \frac{hL}{k_l}
\end{equation}

\begin{equation}
    \label{eq:nusselt_mar}
    Nu = 1.6129\ln (Ma) - 10.183
\end{equation}

\begin{equation}
    \label{eq:marangoni}
    Ma = - \frac{d\sigma}{dT} \frac{L\Delta T \rho C_{p,l}}{\mu k_l}
\end{equation}

\clearpage