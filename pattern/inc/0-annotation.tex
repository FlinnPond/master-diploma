%%% Аннотация %%%
\annotation{Аннотация}

Работа посвящена моделированию процесса селективного плавления 
многокомпонентного материала и изучению изменения концентраций 
его компонета. Это может помочь предсказывать количественный 
состав итогового слава, что позволит судить об итоговых механических 
параметрах, микроструктуре готового изделия, напечатанного методом 
аддитивного производства. Такая возможность актуальна и крайне востребована, 
так как позволяет не проводить большое количество дорогостоящих экспериментов 
и подобрать необходимый сплав исходя из результатов расчёта математической модели.

В работе сформулировано несколько моделей изменения концентрации компонентов 
в сплаве: нульмерная, двух и трёх мерная без гидродинамики, а также модель для 
рачёта концентраций на основе поля скоростей, рассчитанного в программе KiSSAM 
с гидродинамикой.

    


\clearpage