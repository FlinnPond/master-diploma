%%% Аннотация %%%
\annotation{Аннотация}

Работа посвящена моделированию процесса селективного плавления 
многокомпонентного материала и изучению изменения концентраций 
его компонетов. Это может помочь предсказывать количественный 
состав итогового сплава, что позволит судить об итоговых механических 
параметрах, микроструктуре готового изделия, напечатанного методом 
аддитивного производства. Такая возможность актуальна и крайне востребована, 
так как позволяет не проводить большое количество дорогостоящих экспериментов 
и подобрать необходимый сплав исходя из результатов расчёта.

В работе сформулировано несколько моделей изменения концентрации компонентов 
в сплаве, в которых моделируется многокомпонентное испарение:

Нульмерная, в предположении быстрого перемешивания и упрощенной и неизменной формы ванны расплава, 

Трёхмерная, учитывающая нагревание источником, плавление, конвекцию Марангони с помощью эффективного коэффициента и диффузию в расплаве.

Модель, которая берёт за основу данные сложного гидродинамического расчёта и считает конвекцию компонентов в квазистационаре.

С помощью этих моделей исследуется влияние многокомпонетного испарения на итоговый состав сплава.

\clearpage