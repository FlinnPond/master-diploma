\anonsection{Введение}

Аддитивное производство методом лазерного и электронно пучкового спекания 
металлического порошка -- это относительно новое, но довольно перспективное 
направление в области производста деталей и компонентов. Такой метод оказался
крайне актуален и востребован сегодня, что заметно по высокому спросу на 
исследования в его области.

Причины популярности этого метода заключаются в следующих факторах:

\begin{itemize}
    \item Скорость и гибкость производства деталей для малых и средних серий. По сравнению с традиционными методами, где время на производство одной детали сокращается только с ростом количества деталей. 
    \item Дешевизна производста деталей для малых серий. Так же как и со скоростью, цена одной детали снижается только с ростом количества произведённых деталей. К примеру изготовить формы для литья какой-либо сложной детали очень часто обходится гораздо дороже, чем напечатать методом SLM (Selective Laser Melting).
    \item Автоматизация и точность производства. Больше точной работы может быть доверено машине и меньше пространства для ошибок рабочих.
    \item Возможность создавать сложные геометрии, что может быть невозможно при использовании традиционных методов. Кроме того, аддитивное производство позволяет персонализировать изделия.
    \item Меньшее количество отходов и экологических проблем: аддитивные технологии позволяют использовать только нужное количество материала, что снижает количество отходов и экологическую нагрузку.\\
\end{itemize}

Важным и востребованным направлением в этом методе производства является моделирование процесса спекания порошка. Как правило для получения детали с неоходимым качеством, составом, свойствами и без нежелательных артефактов, приходится проводить несколько экспериментов и печатать одну и ту же модель несколько раз с разными параметрами и каждый раз проверять, удовлеторяет ли она необходимым требованиям. На это уходит большое количество времени, сил и ресурсов. Поэтому для удешевления и ускорения процедуры подбора параметров печати используется моделирование. Делая выводы из расчётов, можно кадинально сократить количество долгих и дорогостоящих экспериментов по производству интересующей детали.

Данная работа посвящена моделированию процесса аддитивного производства методом селективного спекания металлического порошка лазером или электронным пучком с целью изучения изменения концентраций компонентов сплава в процессе его плавления. При таком процессе, металл под источником энергии разогревается до очень высоких температур, что приводит к интенсивному испарению и, как следствие, появлению давления отдачи, потере тепла и массы. 

Очень часто для проивзодства детали требуется использовать определённый сплав с определённым количественным составом и определёнными свойствами. При его выборе очень важно учитывать тот факт, что в процессе участвуют компоненты сплава с разными параметрами, что влечет за собой разные скорости испарения разны компонент. Таким образом, итоговая концентрация более летучего компонента всегда будет ниже, чем изначальная, поскольку его испарение происходит более интенсивно.

\clearpage