%%% Моделирование %%%
\section{Моделирование}

Во время селективного спекания порошкового металла под воздействием лазеров или пучков электронов, энергия от них вызывает нагрев частиц порошка на поверхности. 
При этом в точке воздействия источника энергии образуется область, где образуется бассейн рассплавленного металла, который в последествии застывает и соединяется с предыдущими областями, создавая непрерывную деталь.

Моделирование поведения компоненнтов в этой области -- потери от испарения, диффузия, конвекция, плавление и застывание -- позволяет судить об итоговом составе получившейся детали и, как следствие, о её механических свойствах и микроструктуре.

\subsection{Нульмерная модель}

В процессе движения по засыпанному слою порошка, область расплава следует за источником и через какое-то время её размеры перестают расти и до смены направления движения сохраняются около определённого значения. 
Поэтому в данной модели рассматривается поведение концентраций компонентов в предположении бесконечно быстрой диффузии в бассейне расплава неизменной 
прямоугольной формы. На рисунке \ref{fig:zero-model} изоражена геометрическая интерпретация модели.

Уравнение на потоки \ref{eq:zero-general} и формулы для каждого потока выглядят следующим образом \ref{eq:zero-melt} - \ref{eq:lambda}:
\begin{equation}
    \label{eq:zero-general}
    \dot{\rho} = j_{melt} - j_{solid} - j_{evap}
\end{equation}
\begin{equation}
    \label{eq:zero-melt}
    j_{melt} = \lambda \rho
\end{equation}
\begin{equation}
    \label{eq:zer-sol}
    j_{solid} = \lambda \rho_0
\end{equation}
\begin{equation}
    \label{eq:lambda}
    \lambda = \frac{v_{beam}}{L}
\end{equation}

\noindent
здесь $v_{beam}$ -- скорость сканирования, а $\rho$ и $\rho_0$ -- текущая и начальная погонные плотности содержимого бассейна.

Таким образом уравнения \ref{eq:zero-general} - \ref{eq:lambda} в связке с уравнениями для рассчёта испарения \ref{eq:evap} - \ref{eq:cc} формируют систему, описывающую данную нульмерную модель.

Стоит отметить, что данная модель не учитвает многокомпонентности, а уравнение \ref{eq:zero-general} может быть решено аналитически:

\begin{equation}
    \rho = -\frac{j^{evap}}{\lambda} (1 - e^{-\lambda t}) + \rho_0
\end{equation}
\begin{equation}
    \lim_{t\rightarrow \infty } \rho = \rho_0 - \frac{j^{evap}}{\lambda}
\end{equation}


\addimg{zero-model-view.pdf}{0.8}{Нульмерная модель бассейна расплава. Здесь $j_{evap}$ -- поток испарения, $j_{melt}, j_{solid}$ -- потоки приходящего (плавящегося) и уходящего (застывающего) вещества соотвественно, определяемые скоростью движения источника.}{fig:zero-model}



\clearpage