%%% Моделирование %%%
\section{Моделирование}

Во время селективного спекания порошкового металла под воздействием лазеров или пучков электронов, энергия от них вызывает нагрев частиц порошка на поверхности. 
При этом в точке воздействия источника энергии образуется область, где образуется бассейн рассплавленного металла, который в последествии застывает и соединяется с предыдущими областями, создавая непрерывную деталь.

Моделирование поведения компоненнтов в этой области -- потери от испарения, диффузия, конвекция, плавление и застывание -- позволяет судить об итоговом составе получившейся детали и, как следствие, о её механических свойствах и микроструктуре.

\subsection{Нульмерная модель}

В процессе движения по засыпанному слою порошка, область расплава следует за источником и через какое-то время её размеры перестают расти и до смены направления движения сохраняются около определённого значения. 
Поэтому в данной модели рассматривается поведение концентраций компонентов в предположении бесконечно быстрой диффузии в бассейне расплава неизменной 
прямоугольной формы. На рисунке \ref{fig:zero-model} представлно схематическое представление модели.

\addimg{zero-model-view.pdf}{0.8}{Нульмерная модель бассейна расплава. Здесь $j_{evap}$ -- поток испарения, $j_{melt}, j_{solid}$ -- потоки приходящего (плавящегося) и уходящего (застывающего) вещества соотвественно, определяемые скоростью движения источника.}{fig:zero-model}

\clearpage