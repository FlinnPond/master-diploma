%%% Литературный обзор %%%
\section{Литературный обзор}

\subsection{Расчёт испарения}

\subsubsection{Поток испарения}

Для того, чтобы измерить потери компонентов при испарении с поверхности сильно разогретой поверхности металла необходимо оценить поток испарения и конденсации. 
Эти вопросы были исследованы в работе \textit{Knight C. J.} \cite{knight1979theoretical}. 

Максимальный поток испарения вещества оценивается на основании предположения, что поток испарения не может превышать поток конденсации в состоянии равновесия \cite{hertz1882ueber}, на основе чего в работе \cite{langmuir1913vapor} была выведена следующая формула:

\begin{equation}
\label{eq:evap_max}
    j^+ = p_s\sqrt{\frac{m_A}{2\pi k T_s}},
\end{equation} 

где $p_s$ и $T_s$ -- давление и температура непосредственно над поверхностью конденсированной фазы.

\subsubsection{Поток конденсации}

Для оценки потока конденсации в работе \cite{klassen2018simulation} на основе статьи \cite{knight1979theoretical} была получена следующая формула:

\begin{equation}
    \label{eq:evap_coeff}
    \phi = \sqrt{2\pi\gamma} \cdot M_{Kn} \cdot \frac{\rho_{Kn}}{\rho_s} \sqrt{\frac{T_{Kn}}{T_s}},
\end{equation}

где $\gamma$ -- показатель адиабаты материала, $M_{Kn} = \frac{u_{Kn}}{c_{Kn}}$ -- число Маха на выходе из слоя Кнудсена, $u_{Kn}$ и $c_{Kn}$ -- скорость газа и скорость звука в этой же точке соответственно, $\rho_{Kn}, \rho_s$ -- плотность пара на выходе из слоя Кнудсена, плотность пара у поверхности соответственно, $ T_{Kn}, T_s$ -- температура пара на выходе из слоя Кнудсена, температура пара у поверхности соответственно.

\subsubsection{Структура потока испарения}

Неизвестным остаётся число Маха на выходе из слоя Кнудсена. Чтобы понять, что это за параметр и как его можно найти, рассмотрим модель столба пара над испаряющейся поверхностью. 
В ней поток испарения представлется в виде газодинамической задачи Римана о распаде разрыва.  

Картина структуры столба пара представлена на рис. \ref{fig:flowstruct}. Здесь присутствуют обычные компоненты из задачи, такие как ударная волна, контактный разрыв и волна разрежения. Также у основания учитывается наличие слоя Кнудсена, в нём происодит максвеллизация распределения скоростей молекул, вылетевших с поверхности металла. Соотношения для этих переходов выведены Найтом в его статье \cite{knight1979theoretical}.

Слой Кнудсена моделируется исходя из предположения, что распеделение скоростей частиц, вытелтающих с поверхности имеет форму половины от распределения максвелла со средней скоростью равной нулю (рис. \ref{fig:pic1-maxwell-freeflow}).

\addimg{flowstruct.png}{0.8}{Схематическое представление распределения структуры потока пара над поверностью испарения. В режимах: дозвуковом (слева) и сверхзвуком (справа) 
\cite{klassen2018simulation}.}{fig:flowstruct}

\addimg{pic1-maxwell-contflow.png}{0.7}{Распределение вертикальной компоненты скорости частиц на поверхности и на верхней границе слоя Кнудсена(в паре)}{fig:pic1-maxwell-freeflow}

Это предположение позволяет связать параметры с разных сторон слоя Кнудсена \cite{knight1979theoretical}:

\begin{equation}
\label{eq:tempjump}
    \frac{T_{Kn}}{T_s} = \Bigg[ \sqrt{(1 + \pi \left( \frac{\gamma - 1}{\gamma + 1} \frac{M_{Kn}}{2} \right)^2} - \sqrt{\pi} \frac{\gamma - 1}{\gamma + 1} \frac{M_{Kn}}{2} \Bigg]^2
\end{equation}

\begin{multline}
\label{eq:densjump}
    \frac{\rho_{Kn}}{\rho_s} = \sqrt{\frac{T_s}{T_{Kn}}} \Bigg[ \bigg( m^2 + \frac{1}{2} \bigg) \exp\bigg(M_{Kn}^2\bigg)\erfc\bigg(M_{Kn}\bigg) - \frac{M_{Kn}}{\sqrt{\pi}} \bigg] + \\ + \frac{1}{2} \frac{T_s}{T_{Kn}} \bigg[ 1 - \sqrt{\pi}M_{Kn}\exp\bigg(M_{Kn}^2\bigg)\erfc\bigg(M_{Kn}\bigg) \Bigg]
\end{multline}

Уравнение адиабаты Гюгонио, соотношения на контактном разрывеи волне разряжение позволяют связать параметры газа на выходе из слоя Кнудсена и в атмосфере \cite{klassen2018simulation}:

\begin{multline}
\label{eq:rh}
    \frac{p_s}{p_a} = \left( \frac{\rho_{Kn}}{\rho_s} \frac{T_{Kn}}{T_s} \right)^{-1} \cdot \Bigg\{1 + \gamma \cdot M_{Kn} \frac{c_{Kn}}{c_a} \times \\ \times \Bigg[ \frac{\gamma + 1}{4} \cdot M_{Kn} \cdot \frac{c_{Kn}}{c_a} + \sqrt{1 + \left( \frac{\gamma + 1}{4} \cdot M_{Kn} \cdot \frac{c_{Kn}}{c_a} \right)^2} \Bigg] \Bigg\}
\end{multline}

\subsubsection{Давление насыщенного пара}

В полученном уравнении для числа Маха остаётся неизвестным давление насыщенного пара. Его можно оценить разными способами. В работе \cite{klassen2018simulation} используется формура, полученная из уравнения Клапейрона-Клаузиуса:

\begin{multline}
\label{eq:cc}
    p_s = p_{atm}\cdot\exp \Bigg\{ -\frac{m_A L_{vap,0}}{k} \cdot \Bigg[ \frac{1}{T_s} \sqrt{1 - \left( \frac{T_s}{T_{crit}} \right)^2} - \frac{1}{T_{boil}} \sqrt{1 - \left( \frac{T_{boil}}{T_{crit}} \right)^2} + \\ + \frac{1}{T_s} \bigg( \arcsin \left( \frac{T_s}{T_{crit}} \right) - \arcsin\left(\frac{T_{boil}}{T_{crit}} \right)\bigg) \Bigg] \Bigg\}.
\end{multline}

Здесь $(p_{atm}, T_{boil})$ и $(p_{s}, T_{s})$ -- две точки на кривой перехода между газом и жидкостью на фазовой диаграмме материала, $T_{crit}$ -- критическая температура материала.

Также для описания давление насыщенного пара можно использовать уравнение Антони. Оно полу эмпирически описывает зависимость температуры насыщенного пара от его давления для различных веществ и имеет следующий вид:

\begin{equation}
    \log_{10}(p) = A - \frac{B}{C + T}
\end{equation}

где $T$ -- температура насыщенного пара, $A, B, C$ -- параметры вещества, получаемые экспериментально.

\subsubsection{Неидеальность расплава}

При реальной работе со сплавами важно учитывать химические реакции между их компонентами, так как они могут существенно повлиять на процесс испарения.

Для описания многокомпонентности будем считать, что поведение пара во всей структуре потока испарения следует закону Дальтона -- т.е. равно сумме парциальных давлений паров компонентов.

Таким образом формула для потока испаряющихся частиц \ref{eq:evap_max}, \ref{eq:evap_coeff} преобразится в следующий вид \cite{klassen2018simulation}:

\begin{equation}
\label{eq:evap_multicomp_flow}
    j^{\alpha} = \phi^\alpha\chi^\alpha\gamma_{\text{акт}}^\alpha p^{\alpha}_s \cdot \sqrt{\frac{m_A^\alpha}{2\pi k T_s}}
\end{equation}

Здесь $\gamma_{\text{акт}}^\alpha$ -- коэффициент активности компонента $\alpha$, $\chi^\alpha$ -- его мольная доля.

Коэффициенты активности компонентов оцениваются из парциальных энергий Гиббса:

\begin{equation}
    \gamma_{\text{акт}}^\alpha(T_s, \chi) = \exp\left[ \frac{\Delta G^\alpha_\Gamma (T_s, \chi)}{RT_s} \right]
\end{equation}

