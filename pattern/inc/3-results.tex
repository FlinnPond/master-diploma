%%% Результаты %%%
\section{Результаты}

В рамках работы над исследованием влияния испарения на процесс селективного спекания многокомпонентных металлических порошков, стояла задача об описании динамики концентраций в ванне расплава. Этот процесс определяет в какой концентрации компоненты окажутся у поверхности, а значит и какой вклад в поток испарения и в потери, связанные с ним, они будут вносить.
Область у поверхности всегда будет обеднена сильнее, чем более глубокие уровни расплава. Как только легколетучий компонент выходит из глубины, он сразу испаряется. Таким образом процессы переноса играют одну из важнейших ролей в формировании итогового материала и его состава.

Сначала была сформулирована нульмерная модель, которая описывает движение ванны расплава вслед за источником энергии и предполагает неизменную прямоугольную форму ванны расплава и очень быстрое перемешивание. В таком приближении можно судить о максимально возможном обеднении легколетучим компонентом, так как его наличие у поверхности всегда гарантировано предположением о быстром перемешивании. 

По результатам расчётов на этой модели оказалось, что неидеальность расплава сильно влияет на итоговый состав, приводя к ещё более сильному обеднению. Также было отмечено, что концентрации компонентов выходят на режим и не меняются по истечении времени работы модели.

Затем для описания динамики концентраций в ванне расплава была сформулирована новая самосогласованная модель, считающая нагрев от источника, плавление, тепловые потери от испарения и эффективную конвекцию, вызванную эффектом Марангони. Таким образом форму ванны расплава не нужно брать снаружи. Также эта модель учитывает диффузию компонентов, возникающую вследствие испарения.

В этой модели не было выявлено сильного влияния неидеальности. Однако было замечено, что учёт эффектиного конвективного коэффициента приближает форму бассейна расплава к реальным, полученным в экспериментах.

Наконец, в рамках интеграции кода с программой KiSSAM с довольно сложной моделью, где считается гидродинамика и учитывается большое количество эффектов, описание которых в рамках этой работы было бы слишком сложным, была разработана модель, которая берёт за основу данные расчёта из неё, такие как геометрию, температуру, скорости и в считает конвекцию в квазистационаре.

Это привело к сильному (по сравнению с предыдущей моделью) росту потерь, связанных с испарением. 
В дальнейшем возможна дальшнейшая интеграция с кодом программы KiSSAM.

\clearpage